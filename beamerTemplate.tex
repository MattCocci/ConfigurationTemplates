%%%%%%%%%%%%%%%%%%%%%%%%%%%%%%%%%%%%%%%%%%%%%%%%%%%%%%%%%%%%%%%%%%%%%
%% Configure File %%%%%%%%%%%%%%%%%%%%%%%%%%%%%%%%%%%%%%%%%%%%%%%%%%%
%%%%%%%%%%%%%%%%%%%%%%%%%%%%%%%%%%%%%%%%%%%%%%%%%%%%%%%%%%%%%%%%%%%%%

\documentclass{beamer}
%\documentclass[serif]{beamer} % For serif latex font
\usepackage{lmodern}
\usetheme{Boadilla}
%\usetheme{Frankfurt}

% Set up the colors to use with "alerted" text
% --> See the slide "Highlighting Certain Points?" 
\setbeamercolor{dull text}{fg=gray, bg=}
\setbeamercolor{alerted text}{fg=black, bg=}

%%%%%%%%%%%%%%%%%%%%%%%%%%%%%%%%%%%%%%%%%%%%%%%%%%%%%%%%%%%%%%%%%%%%%
%% Fixup Code %%%%%%%%%%%%%%%%%%%%%%%%%%%%%%%%%%%%%%%%%%%%%%%%%%%%%%%
%%%%%%%%%%%%%%%%%%%%%%%%%%%%%%%%%%%%%%%%%%%%%%%%%%%%%%%%%%%%%%%%%%%%%

% This next line removes the institute parens in the
    % footer
\defbeamertemplate*{footline}{my infolines theme}
{
  \leavevmode%
  \hbox{%
  \begin{beamercolorbox}[wd=.333333\paperwidth,ht=2.25ex,dp=1ex,center]{author in head/foot}%
    \usebeamerfont{author in head/foot}\insertshortauthor~~\insertshortinstitute
  \end{beamercolorbox}%
  \begin{beamercolorbox}[wd=.333333\paperwidth,ht=2.25ex,dp=1ex,center]{title in head/foot}%
    \usebeamerfont{title in head/foot}\insertshorttitle
  \end{beamercolorbox}%
  \begin{beamercolorbox}[wd=.333333\paperwidth,ht=2.25ex,dp=1ex,right]{date in head/foot}%
    \usebeamerfont{date in head/foot}\insertshortdate{}\hspace*{2em}
    \insertframenumber{} / \inserttotalframenumber\hspace*{2ex} 
  \end{beamercolorbox}}%
  \vskip0pt%
}



%%%%%%%%%%%%%%%%%%%%%%%%%%%%%%%%%%%%%%%%%%%%%%%%%%%%%%%%%%%%%%%%%%%%%
%% Title Slide %%%%%%%%%%%%%%%%%%%%%%%%%%%%%%%%%%%%%%%%%%%%%%%%%%%%%%
%%%%%%%%%%%%%%%%%%%%%%%%%%%%%%%%%%%%%%%%%%%%%%%%%%%%%%%%%%%%%%%%%%%%%

\begin{document}
\title{Title}   
\author{Author} 
\date{} 

\frame{\titlepage} 


%%%%%%%%%%%%%%%%%%%%%%%%%%%%%%%%%%%%%%%%%%%%%%%%%%%%%%%%%%%%%%%%%%%%%
%% Table of Contents %%%%%%%%%%%%%%%%%%%%%%%%%%%%%%%%%%%%%%%%%%%%%%%%
%%%%%%%%%%%%%%%%%%%%%%%%%%%%%%%%%%%%%%%%%%%%%%%%%%%%%%%%%%%%%%%%%%%%%

% Make table of contents slide
    % Can use sections and subsections before new 
    % frames/slides which will get put in Table of Contents
\frame{ 
    \frametitle{Presentation Structure}
    \tableofcontents
} 


%%%%%%%%%%%%%%%%%%%%%%%%%%%%%%%%%%%%%%%%%%%%%%%%%%%%%%%%%%%%%%%%%%%%%
%% Presentation %%%%%%%%%%%%%%%%%%%%%%%%%%%%%%%%%%%%%%%%%%%%%%%%%%%%%
%%%%%%%%%%%%%%%%%%%%%%%%%%%%%%%%%%%%%%%%%%%%%%%%%%%%%%%%%%%%%%%%%%%%%

\section{Section 1} 
   
    \subsection{Subsection 1.1}
      
	\frame{
	    \frametitle{Slide with Figure}

	    %\begin{figure}
	    %	 \includegraphics[scale=0.75]{timeSeries10,20} 
	    %\end{figure}
	}


   

    \subsection{Subsection 1.2}

	\frame{
	    \frametitle{Slide with Text}
	    
	    You can use lists as usual:
	    \begin{itemize}
		\item You can even \pause add breaks
		\item Anytime you write ``\textbackslash pause'', two slides will
		    be generated \pause
		\item One with everything up to pause only.
		\item Another with everything on the slide, even
		    after pause
	    \end{itemize}
	}


\section{Section 2} 

    \subsection{2.1}

	\frame{
	    \frametitle{Spacing Things Out}

	    Here's an example of adding code to generate 
	    \\
	    \vspace{20pt} 
	    some vertical space
	}


	\frame{
	    \frametitle{Adding Some Color}

        Maybe you want to highlight some text and really make it
        \textcolor{red}{stand out}.
	}

	\frame{
	    \frametitle{Highlighting Certain Points}

        Or maybe, you want to gray out certain text to highlight other
        points. 
        \usebeamercolor{dull text}
        \begin{itemize}
            \item \alert<+>{You do that by defining the color of ``dull''
                text and ``alerted'' text in the preamble}
            \item \alert<+>{Then you set the color to dull text, adding
                ``\textbackslash alert\textless\texttt{+}\textgreater'' 
                tags and enclosing within brackets
                the text to be ``alert.''}
        \end{itemize}
        \pause
        Each ``\textbackslash alert\textless\texttt{+}\textgreater\{text\}'' 
        also adds pauses between the dull and alert
        text.
	}


\end{document}

